% !TeX document-id = {d42653b9-9ea4-48c2-860b-3d6e6e9d8ab0}
% !TeX TXS-program:compile = txs:///pdflatex/[--shell-escape]
\documentclass[12pt]{article} % For LaTeX2e
%\usepackage{neurips_2021}
\usepackage[colorlinks, citecolor={blue}]{hyperref}
\usepackage{url}
\usepackage{amsfonts,amscd,amssymb}
\usepackage{amsthm,amsmath,natbib}
\usepackage{algorithm,algorithmicx,algpseudocode}
\usepackage{bm}
\usepackage{bbm} %bb font numbers
\usepackage[table]{xcolor}
\usepackage{verbatim}
\usepackage{graphicx}
\usepackage{setspace}
\usepackage{natbib}
\usepackage[margin=1in]{geometry}
\usepackage{enumitem}
%\usepackage[nolists]{endfloat}
\usepackage{listings}
\usepackage[textsize=tiny]{todonotes}
\usepackage{tikz}
\usetikzlibrary{shapes.misc}
\usepackage{etoolbox}
\usepackage{appendix}
\usepackage[format=plain,
labelfont={it},
textfont=it]{caption}
\usepackage{subcaption}
\usepackage{wrapfig}
\usepackage{xr}
\usepackage{booktabs}
\usepackage{multirow}
\usepackage{authblk}
\usepackage{mathbbol}



\usetikzlibrary{matrix}
\usetikzlibrary{backgrounds}
\usetikzlibrary{calc}
\usetikzlibrary{arrows,shapes}
\usetikzlibrary{decorations}
\usetikzlibrary{decorations.pathmorphing}
\usetikzlibrary{fit}
\usetikzlibrary{decorations.pathreplacing}

\newtoggle{quickdraw}
\toggletrue{quickdraw} % Uncomment this to render more quickly (non-random)


\definecolor{lightgrey}{rgb}{0.9,0.9,0.9}
\definecolor{darkgreen}{rgb}{0,0.3,0}
%\definecolor{darkred}{rgb}{0.3,0,0}

\definecolorset{rgb}{}{}{darkred,0.8,0,0;darkgreen,0,0.5,0;darkblue,0,0,0.5}

%\doublespacing

\newtheorem{thm}{Theorem}
\newtheorem{lemma}{Lemma}
\newtheorem{prop}{Proposition}
\newtheorem{cor}{Corollary}
\newtheorem{remark}{Remark}
\newtheorem{example}{Example}
\newtheorem{mydef}{Definition}
\newtheorem*{assumption}{Assumption}
\newtheorem{clm}{Claim}

\newcommand{\argmax}{\operatornamewithlimits{arg\,max}}
\newcommand{\argmin}{\operatornamewithlimits{arg\,min}}
\newcommand*{\fplus}{\genfrac{}{}{0pt}{}{}{+}}
\newcommand*{\fdots}{\genfrac{}{}{0pt}{}{}{\cdots}}
\newcommand{\mb}{\mathbf}
\newcommand{\mc}{\mathcal}
\newcommand{\dx}{\mbox{d}}

\renewcommand{\vec}[1]{\mathbf{#1}}
\newcommand{\numTaxa}{N}
\newcommand{\numTraits}{D}
\newcommand{\numDatasets}{M}
%\newcommand{\numLatent}{D}
\newcommand{\taxonIndex}{i}
\newcommand{\traitIndex}{j}
\newcommand{\traitData}{\vec{Y}}
\newcommand{\traitDatum}{y}
\newcommand{\datasetIndex}{m}
\newcommand{\exemplar}{\text{e}}

\newcommand{\sequences}{\vec{S}}
\newcommand{\latentData}{\vec{X}}
\newcommand{\latentdata}{\vec{x}}
\newcommand{\latentDatum}{x}
\newcommand{\phylogeneticParameters}{\boldsymbol{\phi}}
\newcommand{\phylogeny}{{\cal G}}
\newcommand{\tree}{\phylogeny}
%\newcommand{\otherParameters}{\boldsymbol{\
\newcommand{\transpose}{^{t}}

\newcommand{\distanceMatrix}{\mathbf{Y}}
\newcommand{\distance}{y}
\newcommand{\summant}{r}



\newcommand{\cdensity}[2]{\ensuremath{p(#1 \,|\,#2)}}
\newcommand{\density}[1]{\ensuremath{p(#1 )}}

\newcommand{\treeNode}{\nu}

\newcommand{\traitVariance}{\mathbf{\Sigma}}
\newcommand{\nodeIndex}{c}

%\newcommand{\parent}[1]{\mbox{\tiny pa}(#1)}
\newcommand{\parentBig}[1]{\mbox{pa}(#1)}

\newcommand{\sibling}[1]{\mbox{\tiny sib}(#1)}
\newcommand{\siblingBig}[1]{\mbox{sib}(#1)}

\newcommand{\rootMean}{\boldsymbol{\mu}_0}
\newcommand{\rootVarianceScalar}{\tau_0}
\newcommand{\unsequencedVarianceScalar}{\tau_{\exemplar}}
\newcommand{\treeVariance}{\vec{V}_{\tree}}
\newcommand{\hatTreeVariance}{\hat{\vec{V}}_{\tree}}
\newcommand{\mdsSD}{\sigma}
\newcommand{\mdsVariance}{\mdsSD^2}
\newcommand{\residual}{\hat{\traitDatum}}
\newcommand{\modelDistance}{\delta}
\newcommand{\cdf}{\phi}
\newcommand{\normalCDF}[1]{\Phi \left( #1 \right)}

\newcommand{\order}[1]{{\cal O}\hspace{-0.2em}\left( #1 \right)}

\newcommand{\rootNode}{\nu^{\datasetIndex}_{2 \numTaxa_{\datasetIndex} -1 }}
\newcommand{\pathLength}[1]{d(F, #1 )}
\newcommand{\pathLengthNew}[2]{
d_{F}
(
{#1}, {#2}
)
}
\newcommand{\J}{\vec{J}}
\newcommand{\pprime}{^{\prime}}
\newcommand{\otherIndex}{i \pprime}
\def\kronecker{\raisebox{1pt}{\ensuremath{\:\otimes\:}}}

\definecolor{trevorblue}{rgb}{0.330, 0.484, 0.828}
\definecolor{trevoryellow}{rgb}{0.829, 0.680, 0.306}

%%%
%% LaTeX source to reference manuscript changes in revision letters
%%
%% In the manuscript file:
%% 1. Include this file 
%%      %%
%% LaTeX source to reference manuscript changes in revision letters
%%
%% In the manuscript file:
%% 1. Include this file 
%%      %%
%% LaTeX source to reference manuscript changes in revision letters
%%
%% In the manuscript file:
%% 1. Include this file 
%%      \input{make-edits}
%% 2. Define manuscript modifications as
%%      \myedit{UniqueLabel}{Text to appear in manuscript and revision letter}
%%
%% In the revision letter file:
%% 1. Include the automatically updated modifications file
%%      \input{jobname.xtr}
%% 2. Include modified text with
%%      \myeditUniqueLabel
%%
%% Marc A. Suchard
%% 24-Jul-2006
%%

\newwrite\XTR
\AtBeginDocument{\immediate\openout\XTR\jobname.xtr}
\AtEndDocument{\immediate\closeout\XTR}

\newcommand{\myedit}[2]{ % first options is a label, second options is the text
\parbox{0em}{
\shipout\box1{
  \def\mynamea{myedit#1}
  \def\mynameb{\csname \mynamea\endcsname}
\write\XTR{
      \string\newcommand
{\csname myedit#1\endcsname}
}
\write\XTR{
         {``\expandafter\string#2''
         (pg.\string~\thepage)}
}
%\write\XTR{
%    (pg.\string~\thepage)
%}
}
%\hspace*{-1in}
}
%
%\def\thistext{\noexpand#2}
%  \immediate\write\XTR{
%   %     \begin{verbatim}
%        \thistext
%   %     \end{verbatim}
%  }
%\endgroup
%}
%\shipout\vbox{0}
%        \label{\expand\mylabel}
%        \thepage
%  \mylabel
%  \hspace*{-0em}
%  {\bf#2}
#2
}

\newcommand{\myeditblank}[2]{ % first options is a label, second options is the text
\parbox{1em}{
\shipout\box1{
  \def\mynamea{myedit#1}
  \def\mynameb{\csname \mynamea\endcsname}
\write\XTR{
      \string\newcommand
{\csname myedit#1\endcsname}
}
\write\XTR{
         {\expandafter\string#2
         (pg.\string~\thepage)}
}
%\write\XTR{
%    (pg.\string~\thepage)
%}
}
%\hspace*{-1in}
}
%
%\def\thistext{\noexpand#2}
%  \immediate\write\XTR{
%   %     \begin{verbatim}
%        \thistext
%   %     \end{verbatim}
%  }
%\endgroup
%}
%\shipout\vbox{0}
%        \label{\expand\mylabel}
%        \thepage
%  \mylabel
%  \hspace*{-1em}
%  {\bf#2}
}



%\newlength{\strikewidth}
%\newlength{\strikelength}
%\setlength{\strikewidth}{1pt}

%\newcommand{\remove}[1]{
%    \settowidth{\strikelength}{#1}
%    #1\hspace{-\strikelength}
%    \rule[0.5ex]{\strikelength}{\strikewidth}
%}

%\usepackage{ulem}
%\newcommand{\remove}[1]{\sout{#1}}
\newcommand{\remove}[1]{\hspace*{-1em}}

\newcommand{\add}[1]{
%        {\bf #1}
#1%\hspace*{0em}
}


%% 2. Define manuscript modifications as
%%      \myedit{UniqueLabel}{Text to appear in manuscript and revision letter}
%%
%% In the revision letter file:
%% 1. Include the automatically updated modifications file
%%      \input{jobname.xtr}
%% 2. Include modified text with
%%      \myeditUniqueLabel
%%
%% Marc A. Suchard
%% 24-Jul-2006
%%

\newwrite\XTR
\AtBeginDocument{\immediate\openout\XTR\jobname.xtr}
\AtEndDocument{\immediate\closeout\XTR}

\newcommand{\myedit}[2]{ % first options is a label, second options is the text
\parbox{0em}{
\shipout\box1{
  \def\mynamea{myedit#1}
  \def\mynameb{\csname \mynamea\endcsname}
\write\XTR{
      \string\newcommand
{\csname myedit#1\endcsname}
}
\write\XTR{
         {``\expandafter\string#2''
         (pg.\string~\thepage)}
}
%\write\XTR{
%    (pg.\string~\thepage)
%}
}
%\hspace*{-1in}
}
%
%\def\thistext{\noexpand#2}
%  \immediate\write\XTR{
%   %     \begin{verbatim}
%        \thistext
%   %     \end{verbatim}
%  }
%\endgroup
%}
%\shipout\vbox{0}
%        \label{\expand\mylabel}
%        \thepage
%  \mylabel
%  \hspace*{-0em}
%  {\bf#2}
#2
}

\newcommand{\myeditblank}[2]{ % first options is a label, second options is the text
\parbox{1em}{
\shipout\box1{
  \def\mynamea{myedit#1}
  \def\mynameb{\csname \mynamea\endcsname}
\write\XTR{
      \string\newcommand
{\csname myedit#1\endcsname}
}
\write\XTR{
         {\expandafter\string#2
         (pg.\string~\thepage)}
}
%\write\XTR{
%    (pg.\string~\thepage)
%}
}
%\hspace*{-1in}
}
%
%\def\thistext{\noexpand#2}
%  \immediate\write\XTR{
%   %     \begin{verbatim}
%        \thistext
%   %     \end{verbatim}
%  }
%\endgroup
%}
%\shipout\vbox{0}
%        \label{\expand\mylabel}
%        \thepage
%  \mylabel
%  \hspace*{-1em}
%  {\bf#2}
}



%\newlength{\strikewidth}
%\newlength{\strikelength}
%\setlength{\strikewidth}{1pt}

%\newcommand{\remove}[1]{
%    \settowidth{\strikelength}{#1}
%    #1\hspace{-\strikelength}
%    \rule[0.5ex]{\strikelength}{\strikewidth}
%}

%\usepackage{ulem}
%\newcommand{\remove}[1]{\sout{#1}}
\newcommand{\remove}[1]{\hspace*{-1em}}

\newcommand{\add}[1]{
%        {\bf #1}
#1%\hspace*{0em}
}


%% 2. Define manuscript modifications as
%%      \myedit{UniqueLabel}{Text to appear in manuscript and revision letter}
%%
%% In the revision letter file:
%% 1. Include the automatically updated modifications file
%%      \input{jobname.xtr}
%% 2. Include modified text with
%%      \myeditUniqueLabel
%%
%% Marc A. Suchard
%% 24-Jul-2006
%%

\newwrite\XTR
\AtBeginDocument{\immediate\openout\XTR\jobname.xtr}
\AtEndDocument{\immediate\closeout\XTR}

\newcommand{\myedit}[2]{ % first options is a label, second options is the text
\parbox{0em}{
\shipout\box1{
  \def\mynamea{myedit#1}
  \def\mynameb{\csname \mynamea\endcsname}
\write\XTR{
      \string\newcommand
{\csname myedit#1\endcsname}
}
\write\XTR{
         {``\expandafter\string#2''
         (pg.\string~\thepage)}
}
%\write\XTR{
%    (pg.\string~\thepage)
%}
}
%\hspace*{-1in}
}
%
%\def\thistext{\noexpand#2}
%  \immediate\write\XTR{
%   %     \begin{verbatim}
%        \thistext
%   %     \end{verbatim}
%  }
%\endgroup
%}
%\shipout\vbox{0}
%        \label{\expand\mylabel}
%        \thepage
%  \mylabel
%  \hspace*{-0em}
%  {\bf#2}
#2
}

\newcommand{\myeditblank}[2]{ % first options is a label, second options is the text
\parbox{1em}{
\shipout\box1{
  \def\mynamea{myedit#1}
  \def\mynameb{\csname \mynamea\endcsname}
\write\XTR{
      \string\newcommand
{\csname myedit#1\endcsname}
}
\write\XTR{
         {\expandafter\string#2
         (pg.\string~\thepage)}
}
%\write\XTR{
%    (pg.\string~\thepage)
%}
}
%\hspace*{-1in}
}
%
%\def\thistext{\noexpand#2}
%  \immediate\write\XTR{
%   %     \begin{verbatim}
%        \thistext
%   %     \end{verbatim}
%  }
%\endgroup
%}
%\shipout\vbox{0}
%        \label{\expand\mylabel}
%        \thepage
%  \mylabel
%  \hspace*{-1em}
%  {\bf#2}
}



%\newlength{\strikewidth}
%\newlength{\strikelength}
%\setlength{\strikewidth}{1pt}

%\newcommand{\remove}[1]{
%    \settowidth{\strikelength}{#1}
%    #1\hspace{-\strikelength}
%    \rule[0.5ex]{\strikelength}{\strikewidth}
%}

%\usepackage{ulem}
%\newcommand{\remove}[1]{\sout{#1}}
\newcommand{\remove}[1]{\hspace*{-1em}}

\newcommand{\add}[1]{
%        {\bf #1}
#1%\hspace*{0em}
}


%\makeatletter
%\def\title@font{\Huge}
%\let\ltx@maketitle\@maketitle
%\def\@maketitle{\bgroup%
%	\let\ltx@title\@title%
%	\def\@title{\resizebox{\textwidth}{!}{%
%			\mbox{\title@font\ltx@title}%
%	}}%
%	\ltx@maketitle%
%	\egroup}
%\makeatother


\title{Quantum parallel Markov chain Monte Carlo}
\date{}



\author{Andrew J.~Holbrook}


\affil{UCLA Biostatistics}





\renewcommand\Authands{ and }


\graphicspath{{figures/}}

\begin{document}


\maketitle




\begin{abstract}

We propose a novel quantum computing strategy for \emph{parallel MCMC} algorithms that generate multiple proposals at each step. This strategy makes parallel MCMC amenable to quantum parallelization by untangling the generalized accept-reject step in a way that combines the Gumbel-max trick with the recent simplicial sampler proposal mechanism.  Next, we embed this classical routine within a well-known extension of Grover's quantum search algorithm to select our next Markov chain state.  Letting $K$ denote the number of proposals in a single MCMC iteration, the combined strategy reduces the number of target evaluations required from $\order{K}$ to $\order{K^{1/2}}$.  Notably, these speedups do not preclude classical parallelization of target evaluations and proposal generations.



\end{abstract}


\section{Introduction}

\newcommand{\ttheta}{\boldsymbol{\theta}}
\newcommand{\dd}{\mbox{d}}

Parallel MCMC techniques use multiple proposals to obtain efficiency gains over MCMC algorithms such as Metropolis-Hastings \citep{metropolis1953equation,hastings1970monte} and its progeny that use only a single proposal.  \citet{neal2003markov} develops efficient MCMC transitions for inferring the states of hidden Markov models by proposing a `pool' of candidate states and using dynamic programming to select from among them.  \citet{tjelmeland2004using} considers inference in the general setting and shows how to maintain detailed balance for an arbitrary number (say, $K$) of proposals.  Consider a probability distribution $\pi(\dd\ttheta)$ defined on $\mathbb{R}^D$ that admits a probability density $\pi(\ttheta)$ with respect to the Lebesgue measure, i.e., $\pi(\dd \ttheta)=:\pi(\ttheta)\dd\ttheta$. To generate samples from the target distribution $\pi$, we craft a kernel $P(\ttheta_0,\dd \ttheta)$ that satisfies
\begin{align}\label{eq:fixes}
\pi(A) = \int \pi(\dd \ttheta_0) P(\ttheta_0,A) \, 
\end{align} 
for all  $A \subset \mathbb{R}^D$ for which $\pi(A) > 0$.  Letting $\ttheta_0$ denote the current state of the Markov chain, \citet{tjelmeland2004using} proposes sampling from such a kernel $P(\ttheta_0,\cdot)$ by drawing $K$ proposals $\ttheta_1,\dots,\ttheta_K$ from a distribution $Q(\ttheta_0,\dd \ttheta) =: q(\ttheta_0,\ttheta)\dd \ttheta$ and selecting the next Markov chain state from among the current and proposed states with probabilities
\begin{align}\label{eq:probs}
\pi_j = \frac{\pi(\ttheta_j) \prod_{k \neq j} q(\ttheta_j,\ttheta_k)}{\sum_{i=0}^K \pi(\ttheta_i) \prod_{k \neq i} q(\ttheta_i,\ttheta_k)} \, , \quad j=0,\dots,K \, .
\end{align}
\citet{tjelmeland2004using} shows that the kernel $P(\ttheta_0,\cdot)$ constructed in such a manner maintains detailed balance and hence satisfies Equation \eqref{eq:fixes}.  Others have since built on this work, developing parallel MCMC methods that generate or recycle multiple proposals \citep{frenkel2004speed,delmas2009does,calderhead2014general,yang2018parallelizable,luo2019multiple,schwedes2021rao,holbrook2021generating}.  
These developments demonstrate the ability of parallel MCMC methods to alleviate inferential challenges such as multimodality and to deliver performance gains over single-proposal competitors as measured by reduced autocorrelation between samples.

But the real promise and power of parallel MCMC comes from its natural parallelizability \citep{calderhead2014general}.  Contemporary hardware design emphasizes architectures that enable execution of multiple mathematical operations simultaneously. Parallel MCMC techniques stand to leverage technological developments that keep modern computation on track with Moore's Law, which predicts that processing power doubles every two years.  For example, the algorithm of \citet{tjelmeland2004using} generates $K$ conditionally independent proposals and then evaluates the probabilities of Equation \eqref{eq:probs}.  One may parallelize the proposal generation step using parallel pseudorandom number generators (PRNG) such as those advanced in \citet{salmon2011parallel}. The computational complexity of the target evaluations $\pi(\ttheta_j)$ is linear in the number of proposals. This presents a significant burden when $\pi(\cdot)$ is computationally expensive, e.g., in big data settings, but evaluation of the target density over the $K$ proposals is again a naturally parallelizable task.  Moreover, widely available machine learning software such as \textsc{TensorFlow} allows users to easily parallelize both random number generation and target evaluations on general purpose graphics processing units (GPU) \citep{lao2020tfp}. Finally, the pairwise evaluations of the proposal density $q(\cdot,\cdot)$ in Equation \eqref{eq:probs} are $\order{K^2}$, but \citet{massive} demonstrate the natural parallelizability of such pairwise operations, obtaining multiple orders-of-magnitude speedups with contemporary GPUs.

While parallel MCMC algorithms are increasingly well-suited for developing many-core computational architectures, there are trade-offs that need to be taken into account when choosing how to allocate computational resources.  On one end of the spectrum, \citet{gelman1992inference} demonstrate the usefulness of generating, combining and comparing multiple independent Markov chains that target the same distribution, and one may perform this embarrassingly parallel task by assigning the operations for each individual chain to a separate central processing unit (CPU) core or GPU work-group.  In this multi-chain context, simultaneously running multiple parallel MCMC chains could limit resources available for the within-chain parallelization described above.  On the other end of the spectrum, one may find it useful to allocate resources to overcome computational bottlenecks within a single Markov chain that uses a traditional accept-reject step. In big data contexts, \citet{holbrook2021viral,massive,holbrook2021scalable} use multi- and many-core processing to accelerate log-likelihood and log-likelihood gradient calculations within single Metropolis-Hastings and Hamiltonian Monte Carlo \citep{neal2011mcmc} generated chains.  This big data strategy might again limit resources available for parallelization across proposals and target evaluations described above.

\emph{In the presence of these trade-offs, we seek to enjoy the benefits of parallel MCMC while (\textcolor{red}{a}) limiting the total amount of computing necessary and (\textcolor{red}{b})  retaining parallel computing resources for computational bottlenecks such as proposal generations and target evaluations. Here, we assert that established quantum algorithms can help achieve these twin goals.}

A quantum computer acts on complex data vectors of magnitude 1 called qubits with gates that are mathematically equivalent to unitary operators \citep{nielsen2002quantum}.  Assuming that engineers overcome the tremendous difficulties involved in building a practical quantum computer (where practicality entails simultaneous use of many quantum gates with little additional noise), 21st century statisticians might have access to quadratic or even exponential speedups for extremely specific statistical tasks.  We are particularly interested in the following four quantum algorithms: quantum search \citep{grover1996fast}, or finding a single 1 amid a collection of 0s, only requires $\order{\sqrt{N}}$ queries, delivering a quadratic speedup over classical search; quantum counting \citep{boyer1998tight}, or finding the number of 1s amid a collection of 0s, only requires $\order{\sqrt{N/M}}$ (where $M$ is the number of 1s) and could be useful for generating p-values within Monte Carlo simulation from a null distribution; to obtain the gradient of a function (e.g., the log-likelihood for Fisher scoring or HMC) with a quantum computer, one only needs to evaluate the function once \citep{jordan2005fast} as opposed to $\order(P)$ times for numerical differentiation, and there is nothing stopping the statistician from using, say, a GPU for this single function call; finally, the HHL algorithm \citep{harrow2009quantum} obtains the scalar value 

\section{Quantum search and quantum minimization}


\section{Parallel MCMC}

\subsection{Untangling computations}

\begin{enumerate}
	\item Challenge: proposal terms in acceptance
	\item Challenge: normalization/marginalization
\end{enumerate}

\subsubsection{Simplicial sampler}

\subsubsection{Gumbel-max}

\section{Quantum optimization}


\section{Algorithms}

\section{Discussion}

\bibliographystyle{sysbio}
\bibliography{refs}

\appendix




%%%%%%%%%%%%%%%%%%%%%%%%%%%%%%%%%%%%%%%%%%%%%%%%%%%%%%%%%%%%

\end{document}
